\documentclass[11pt]{report}
% ⚠️ DO NOT TOUCH ————————————
% * include and settings
% ✏️ note : include should be used in document environment → use of input here
% & definition
\newcounter{code}
\newcounter{proof}
% & value
\setcounter{code}{1}
\setcounter{proof}{0}
% ~ if statements
\ifnum\value{code}=1
\fi
\ifnum\value{proof}=1
\fi
\usepackage[utf8]{inputenc}
\usepackage[T1]{fontenc}
\usepackage{graphicx}
\usepackage{amsmath,amssymb}
\usepackage{hyperref}
\usepackage[french]{babel}
\usepackage{url}
% tables
\usepackage[table,xcdraw]{xcolor}
\usepackage{array}
\usepackage{booktabs}
\usepackage{tabularx}
% 
\usepackage{pgfplots}
\usepackage{stmaryrd}
\usepackage{mathtools}
% 
\usepackage{algorithm2e}
\usepackage[bottom]{footmisc}
% ⚠️ incompatible avec d autres packages
% \usepackage{MnSymbol}
\usepackage{comment}

\usepackage{float}
\usepackage{multirow}
\usepackage[top=1.5cm,bottom=1.5cm,margin=1.5cm]{geometry}
\usepackage{tikz}
\usepackage{tikz-cd}

%%%%%%% left bars
%% https://latex.org/forum/viewtopic.php?t=5580
\usepackage{framed}

% Multi column
\usepackage{multicol}

%    BOXES
\usepackage{awesomebox}
% code syntax highlight
% use pdflatex --shell-escape
% latexmk -pdf -time -silent -pvc -bibtex


% https://tex.stackexchange.com/questions/39147/scale-image-to-page-width
\usepackage{changepage}
% \usepackage{doc}

% > TO-DO LIST environment
% https://tex.stackexchange.com/questions/247681/how-to-create-checkbox-todo-list
\usepackage{enumitem}
\newlist{todolist}{itemize}{2}
\setlist[todolist]{label=$\square$}
\usepackage{pifont}
\newcommand{\cmark}{\ding{51}}%
\newcommand{\xmark}{\ding{55}}%
\newcommand{\checked}{\rlap{$\square$}{\raisebox{2pt}{\large\hspace{1pt}\cmark}}%
	\hspace{-2.5pt}}
\newcommand{\crossed}{\rlap{$\square$}{\large\hspace{1pt}\xmark}}
% TODO : ajouter a latex-template
\newlist{circledenum}{enumerate}{1}
\setlist[circledenum]{label=\protect\circled{\arabic*}}


\sffamily
\renewcommand{\familydefault}{\sfdefault}
% \usepackage{helvet}
\usepackage{avant}

\usepackage{dsfont}  % The distribution package name is doublestroke in TeX Live for the LaTeX package dsfont.

% emojis
\usepackage{fontawesome5}
\usepackage{mfirstuc}

\usepackage{minitoc}


\usepackage[utf8]{inputenc}
\usepackage[T1]{fontenc}
\usepackage{graphicx}
\usepackage{amsmath,amssymb}
\usepackage{hyperref}
\usepackage[french]{babel}
\usepackage{url}
% tables
\usepackage[table,xcdraw]{xcolor}
\usepackage{array}
\usepackage{booktabs}
\usepackage{tabularx}
% 
\usepackage{pgfplots}
\usepackage{stmaryrd}
\usepackage{mathtools}
% 
\usepackage{algorithm2e}
\usepackage[bottom]{footmisc}
% ⚠️ incompatible avec d autres packages
% \usepackage{MnSymbol}
\usepackage{comment}

\usepackage{float}
\usepackage{multirow}
\usepackage[top=1.5cm,bottom=1.5cm,margin=1.5cm]{geometry}
\usepackage{tikz}
\usepackage{tikz-cd}

%%%%%%% left bars
%% https://latex.org/forum/viewtopic.php?t=5580
\usepackage{framed}

% Multi column
\usepackage{multicol}

%    BOXES
\usepackage{awesomebox}
% code syntax highlight
% use pdflatex --shell-escape
% latexmk -pdf -time -silent -pvc -bibtex


% https://tex.stackexchange.com/questions/39147/scale-image-to-page-width
\usepackage{changepage}
% \usepackage{doc}

% > TO-DO LIST environment
% https://tex.stackexchange.com/questions/247681/how-to-create-checkbox-todo-list
\usepackage{enumitem}
\newlist{todolist}{itemize}{2}
\setlist[todolist]{label=$\square$}
\usepackage{pifont}
\newcommand{\cmark}{\ding{51}}%
\newcommand{\xmark}{\ding{55}}%
\newcommand{\checked}{\rlap{$\square$}{\raisebox{2pt}{\large\hspace{1pt}\cmark}}%
	\hspace{-2.5pt}}
\newcommand{\crossed}{\rlap{$\square$}{\large\hspace{1pt}\xmark}}
% TODO : ajouter a latex-template
\newlist{circledenum}{enumerate}{1}
\setlist[circledenum]{label=\protect\circled{\arabic*}}


\sffamily
\renewcommand{\familydefault}{\sfdefault}
% \usepackage{helvet}
\usepackage{avant}

\usepackage{dsfont}  % The distribution package name is doublestroke in TeX Live for the LaTeX package dsfont.

% emojis
\usepackage{fontawesome5}
\usepackage{mfirstuc}

\usepackage{minitoc}


% ✅ only text
\newcommand{\auteur}{Allemand Instable}
\newcommand{\articletitle}{TITLE}
\newcommand{\customdate}{22 Mai 2022}
\newcommand{\projecttitle}{PROJECT TITLE}
\newcommand{\projectdescription}{Project Description}
% ⚠️ HREF package
\newcommand{\mail}{\href{mailto:redacted@gmail.com}{redacted@gmail.com}}
\newcommand{\github}{\href{https://github.com/allemand-instable/}{allemand-instable}}
\newcommand{\githubissues}{\href{https://github.com/allemand-instable/LaTeX-Template/issues}{LaTeX-Template/issues}}

\bibliographystyle{plain}
% ~ MINTED

% \newcommand{\setcoderender}[1]{%
% 	\newcounter{coderenderbool}%
% 	\setcounter{coderenderbool}{#1}%
% }

% \newcommand{\ifcode}[1]{%
% 	\ifnum1=\value{coderenderbool}%
% 	{%
% 		{#1}%
% 	}%
% 	\else
% 	{%
% 		%
% 	}%
% }


\ifnum\value{code}=1
	\usepackage[cache=false]{minted}

	\usemintedstyle{one-dark}
\fi

% document
\usepackage{pdfpages}
% TODO : ajouter à latex-template
% ! notations —————————————————
% TODO : ajouter à latex-template
\newcommand{\dash}{\rule[1pt]{0.5cm}{0.02cm}}
% ~ Noyaux
\newcommand{\Kh}[1]{K_h\left( \, #1 \, \right)}
\newcommand{\Khs}[1]{K_h^{\star}\left( \, #1 \, \right)}
\newcommand{\Khsd}[2]{K_h^{\star}\left( \, #1, #2 \, \right)}
\newcommand{\Khdev}[1]{\frac 1 h K\left( \frac{#1}{h} \right)}
\newcommand{\Kdeconvo}[1]{K^{[\textsf{deconv}]}_h\left( \, #1 \, \right)}
% ~ convolution
\newcommand{\convo}[2]{ \bigl[\,  #1  *  #2 \, \bigr] }
% ~ observations bruitées
\newcommand{\xtei}{\widetilde X^{*[\,i\,]}}
\newcommand{\yti}{\widetilde Y^{[\,i\,]}}
\newcommand{\zei}{Z^{*[\,i\,]}}
\newcommand{\xei}{X^{*[\,i\,]}}
\newcommand{\obsi}[1]{{#1}^{[\,i\,]}}
\newcommand{\yi}{\obsi{Y}}
\newcommand{\zi}{\obsi{Z}}
\newcommand{\vi}{\obsi{V}}
\newcommand{\ui}{\obsi{U}}

% ! —————————————————————————
% ✅ only text
\newcommand{\auteur}{Allemand Instable}
\newcommand{\articletitle}{TITLE}
\newcommand{\customdate}{22 Mai 2022}
\newcommand{\projecttitle}{PROJECT TITLE}
\newcommand{\projectdescription}{Project Description}
% ⚠️ HREF package
\newcommand{\mail}{\href{mailto:redacted@gmail.com}{redacted@gmail.com}}
\newcommand{\github}{\href{https://github.com/allemand-instable/}{allemand-instable}}
\newcommand{\githubissues}{\href{https://github.com/allemand-instable/LaTeX-Template/issues}{LaTeX-Template/issues}}

% ⚠️ ————————————————————————

% * global commands
% ⚙️ INCLUDE COMMAND
%% ================= commandes ========================== 
%%                   -globales

\usepackage[utf8]{inputenc}
\usepackage[T1]{fontenc}
\usepackage{graphicx}
\usepackage{amsmath,amssymb}
\usepackage{hyperref}
\usepackage[french]{babel}
\usepackage{url}
% tables
\usepackage[table,xcdraw]{xcolor}
\usepackage{array}
\usepackage{booktabs}
\usepackage{tabularx}
% 
\usepackage{pgfplots}
\usepackage{stmaryrd}
\usepackage{mathtools}
% 
\usepackage{algorithm2e}
\usepackage[bottom]{footmisc}
% ⚠️ incompatible avec d autres packages
% \usepackage{MnSymbol}
\usepackage{comment}

\usepackage{float}
\usepackage{multirow}
\usepackage[top=1.5cm,bottom=1.5cm,margin=1.5cm]{geometry}
\usepackage{tikz}
\usepackage{tikz-cd}

%%%%%%% left bars
%% https://latex.org/forum/viewtopic.php?t=5580
\usepackage{framed}

% Multi column
\usepackage{multicol}

%    BOXES
\usepackage{awesomebox}
% code syntax highlight
% use pdflatex --shell-escape
% latexmk -pdf -time -silent -pvc -bibtex


% https://tex.stackexchange.com/questions/39147/scale-image-to-page-width
\usepackage{changepage}
% \usepackage{doc}

% > TO-DO LIST environment
% https://tex.stackexchange.com/questions/247681/how-to-create-checkbox-todo-list
\usepackage{enumitem}
\newlist{todolist}{itemize}{2}
\setlist[todolist]{label=$\square$}
\usepackage{pifont}
\newcommand{\cmark}{\ding{51}}%
\newcommand{\xmark}{\ding{55}}%
\newcommand{\checked}{\rlap{$\square$}{\raisebox{2pt}{\large\hspace{1pt}\cmark}}%
	\hspace{-2.5pt}}
\newcommand{\crossed}{\rlap{$\square$}{\large\hspace{1pt}\xmark}}
% TODO : ajouter a latex-template
\newlist{circledenum}{enumerate}{1}
\setlist[circledenum]{label=\protect\circled{\arabic*}}


\sffamily
\renewcommand{\familydefault}{\sfdefault}
% \usepackage{helvet}
\usepackage{avant}

\usepackage{dsfont}  % The distribution package name is doublestroke in TeX Live for the LaTeX package dsfont.

% emojis
\usepackage{fontawesome5}
\usepackage{mfirstuc}

\usepackage{minitoc}



\usepackage[utf8]{inputenc}
\usepackage[T1]{fontenc}
\usepackage{graphicx}
\usepackage{amsmath,amssymb}
\usepackage{hyperref}
\usepackage[french]{babel}
\usepackage{url}
% tables
\usepackage[table,xcdraw]{xcolor}
\usepackage{array}
\usepackage{booktabs}
\usepackage{tabularx}
% 
\usepackage{pgfplots}
\usepackage{stmaryrd}
\usepackage{mathtools}
% 
\usepackage{algorithm2e}
\usepackage[bottom]{footmisc}
% ⚠️ incompatible avec d autres packages
% \usepackage{MnSymbol}
\usepackage{comment}

\usepackage{float}
\usepackage{multirow}
\usepackage[top=1.5cm,bottom=1.5cm,margin=1.5cm]{geometry}
\usepackage{tikz}
\usepackage{tikz-cd}

%%%%%%% left bars
%% https://latex.org/forum/viewtopic.php?t=5580
\usepackage{framed}

% Multi column
\usepackage{multicol}

%    BOXES
\usepackage{awesomebox}
% code syntax highlight
% use pdflatex --shell-escape
% latexmk -pdf -time -silent -pvc -bibtex


% https://tex.stackexchange.com/questions/39147/scale-image-to-page-width
\usepackage{changepage}
% \usepackage{doc}

% > TO-DO LIST environment
% https://tex.stackexchange.com/questions/247681/how-to-create-checkbox-todo-list
\usepackage{enumitem}
\newlist{todolist}{itemize}{2}
\setlist[todolist]{label=$\square$}
\usepackage{pifont}
\newcommand{\cmark}{\ding{51}}%
\newcommand{\xmark}{\ding{55}}%
\newcommand{\checked}{\rlap{$\square$}{\raisebox{2pt}{\large\hspace{1pt}\cmark}}%
	\hspace{-2.5pt}}
\newcommand{\crossed}{\rlap{$\square$}{\large\hspace{1pt}\xmark}}
% TODO : ajouter a latex-template
\newlist{circledenum}{enumerate}{1}
\setlist[circledenum]{label=\protect\circled{\arabic*}}


\sffamily
\renewcommand{\familydefault}{\sfdefault}
% \usepackage{helvet}
\usepackage{avant}

\usepackage{dsfont}  % The distribution package name is doublestroke in TeX Live for the LaTeX package dsfont.

% emojis
\usepackage{fontawesome5}
\usepackage{mfirstuc}

\usepackage{minitoc}



\usepackage[utf8]{inputenc}
\usepackage[T1]{fontenc}
\usepackage{graphicx}
\usepackage{amsmath,amssymb}
\usepackage{hyperref}
\usepackage[french]{babel}
\usepackage{url}
% tables
\usepackage[table,xcdraw]{xcolor}
\usepackage{array}
\usepackage{booktabs}
\usepackage{tabularx}
% 
\usepackage{pgfplots}
\usepackage{stmaryrd}
\usepackage{mathtools}
% 
\usepackage{algorithm2e}
\usepackage[bottom]{footmisc}
% ⚠️ incompatible avec d autres packages
% \usepackage{MnSymbol}
\usepackage{comment}

\usepackage{float}
\usepackage{multirow}
\usepackage[top=1.5cm,bottom=1.5cm,margin=1.5cm]{geometry}
\usepackage{tikz}
\usepackage{tikz-cd}

%%%%%%% left bars
%% https://latex.org/forum/viewtopic.php?t=5580
\usepackage{framed}

% Multi column
\usepackage{multicol}

%    BOXES
\usepackage{awesomebox}
% code syntax highlight
% use pdflatex --shell-escape
% latexmk -pdf -time -silent -pvc -bibtex


% https://tex.stackexchange.com/questions/39147/scale-image-to-page-width
\usepackage{changepage}
% \usepackage{doc}

% > TO-DO LIST environment
% https://tex.stackexchange.com/questions/247681/how-to-create-checkbox-todo-list
\usepackage{enumitem}
\newlist{todolist}{itemize}{2}
\setlist[todolist]{label=$\square$}
\usepackage{pifont}
\newcommand{\cmark}{\ding{51}}%
\newcommand{\xmark}{\ding{55}}%
\newcommand{\checked}{\rlap{$\square$}{\raisebox{2pt}{\large\hspace{1pt}\cmark}}%
	\hspace{-2.5pt}}
\newcommand{\crossed}{\rlap{$\square$}{\large\hspace{1pt}\xmark}}
% TODO : ajouter a latex-template
\newlist{circledenum}{enumerate}{1}
\setlist[circledenum]{label=\protect\circled{\arabic*}}


\sffamily
\renewcommand{\familydefault}{\sfdefault}
% \usepackage{helvet}
\usepackage{avant}

\usepackage{dsfont}  % The distribution package name is doublestroke in TeX Live for the LaTeX package dsfont.

% emojis
\usepackage{fontawesome5}
\usepackage{mfirstuc}

\usepackage{minitoc}



\input{include/commands/links.tex}

\usepackage[utf8]{inputenc}
\usepackage[T1]{fontenc}
\usepackage{graphicx}
\usepackage{amsmath,amssymb}
\usepackage{hyperref}
\usepackage[french]{babel}
\usepackage{url}
% tables
\usepackage[table,xcdraw]{xcolor}
\usepackage{array}
\usepackage{booktabs}
\usepackage{tabularx}
% 
\usepackage{pgfplots}
\usepackage{stmaryrd}
\usepackage{mathtools}
% 
\usepackage{algorithm2e}
\usepackage[bottom]{footmisc}
% ⚠️ incompatible avec d autres packages
% \usepackage{MnSymbol}
\usepackage{comment}

\usepackage{float}
\usepackage{multirow}
\usepackage[top=1.5cm,bottom=1.5cm,margin=1.5cm]{geometry}
\usepackage{tikz}
\usepackage{tikz-cd}

%%%%%%% left bars
%% https://latex.org/forum/viewtopic.php?t=5580
\usepackage{framed}

% Multi column
\usepackage{multicol}

%    BOXES
\usepackage{awesomebox}
% code syntax highlight
% use pdflatex --shell-escape
% latexmk -pdf -time -silent -pvc -bibtex


% https://tex.stackexchange.com/questions/39147/scale-image-to-page-width
\usepackage{changepage}
% \usepackage{doc}

% > TO-DO LIST environment
% https://tex.stackexchange.com/questions/247681/how-to-create-checkbox-todo-list
\usepackage{enumitem}
\newlist{todolist}{itemize}{2}
\setlist[todolist]{label=$\square$}
\usepackage{pifont}
\newcommand{\cmark}{\ding{51}}%
\newcommand{\xmark}{\ding{55}}%
\newcommand{\checked}{\rlap{$\square$}{\raisebox{2pt}{\large\hspace{1pt}\cmark}}%
	\hspace{-2.5pt}}
\newcommand{\crossed}{\rlap{$\square$}{\large\hspace{1pt}\xmark}}
% TODO : ajouter a latex-template
\newlist{circledenum}{enumerate}{1}
\setlist[circledenum]{label=\protect\circled{\arabic*}}


\sffamily
\renewcommand{\familydefault}{\sfdefault}
% \usepackage{helvet}
\usepackage{avant}

\usepackage{dsfont}  % The distribution package name is doublestroke in TeX Live for the LaTeX package dsfont.

% emojis
\usepackage{fontawesome5}
\usepackage{mfirstuc}

\usepackage{minitoc}



% ~ proof

\newcommand{\insertproof}[1]{%
	\ifnum\value{proofrenderbool}=1%
		{%
		\input{#1}%
		}%
	\else
		{%
			%
		}%
	\fi
}

% ~ MINTED

\newcommand{\insertcode}[1]{%
	\ifnum\value{code}=1%
		{%
		\input{#1}%
		}%
	\else
		{%
			%
		}%
	\fi
}


\begin{document}

	\tableofcontents

	\chapter{Documentation}
	
	% 🧠 why ?
	% si l environnement minted n est pas appelé au moins une fois (executé), alors minted ne génère pas le fichier aux de style .pygstyle nécesaaire pour les inlines
	\begin{phantom}
		\begin{minted}{latex}
		\end{minted}
	\end{phantom}

	\section{Packages \& Dependencies}

	\noindent\begin{tabularx}{\linewidth}{|X|X|}
		\hline
		\textbf{Package}        & \textbf{Description}                                                                                                   \\
		\hline
		\verb|inputenc|         & Allows the user to input accented characters directly from the keyboard, without having to use special commands.       \\
		\hline
		\verb|fontenc|          & Allows the user to select font encodings.                                                                              \\
		\hline
		\verb|graphicx|         & Provides a key-value interface for optional arguments to the \verb|\includegraphics| command.                          \\
		\hline
		\verb|amsmath, amssymb| & Provides various mathematical symbols and environments.                                                                \\
		\hline
		\verb|hyperref|         & Provides extensive support for hypertext in LaTeX.                                                                     \\
		\hline
		\verb|babel|            & Provides internationalization for LaTeX.                                                                               \\
		\hline
		\verb|url|              & Provides commands for typesetting URLs.                                                                                \\
		\hline
		\verb|xcolor|           & Provides easy driver-independent access to several kinds of color tints, shades, tones, and mixes of arbitrary colors. \\
		\hline
		\verb|array|            & Provides an extended implementation of the array and tabular environments.                                             \\
		\hline
		\verb|booktabs|         & Provides commands to enhance the quality of tables.                                                                    \\
		\hline
		\verb|tabularx|         & Provides an environment for tables that automatically adjusts the width of columns to achieve a specified total width. \\
		\hline
		\verb|pgfplots|         & Provides tools to generate plots and diagrams.                                                                         \\
		\hline
		\verb|stmaryrd|         & Provides various symbols for mathematical logic.                                                                       \\
		\hline
		\verb|mathtools|        & Provides various tools to enhance the appearance and functionality of mathematical formulas.                           \\
		\hline
		\verb|algorithm2e|      & Provides an environment for writing algorithms in LaTeX.                                                               \\
	\end{tabularx}

	\noindent\begin{tabularx}{\linewidth}{|X|X|}
		% \hline
		\verb|footmisc|         & Provides several options for customizing footnotes.                                                                    \\
		\hline
		\verb|comment|          & Provides an environment for commenting out sections of text.                                                           \\
		\hline
		\verb|mfirstuc|         & Provides commands for capitalizing the first letter of a word.                                                         \\
		\hline
		\verb|float|            & Provides improved interface for floating objects such as figures and tables.                                           \\
		\hline
		\verb|multirow|         & Provides commands for multi-row cells in tables.                                                                       \\
		\hline
		\verb|geometry|         & Provides an easy and flexible interface to customize page layout.                                                      \\
		\hline
		\verb|tikz|             & Provides a powerful tool to create graphics in LaTeX.                                                                  \\
		\hline
		\verb|tikz-cd|          & Provides a specialized tool for creating commutative diagrams.                                                         \\
		\hline
		\verb|framed|           & Provides an environment for creating framed boxes.                                                                     \\
		\hline
		\verb|multicol|         & Provides an environment for multicolumn typesetting.                                                                   \\
		\hline
		\verb|awesomebox|       & Provides various types of colored boxes.                                                                               \\
		\hline
		\verb|changepage|       & Provides commands to change the page layout in the middle of a document.                                               \\
		\hline
	\end{tabularx}

	\info{
		\smallskip\centering The descriptions have been made using GPT (because it's boring and long), some of the descriptions might not be fully accurate
	}

	\section{Commands}

	\subsection{Commands Description}

	\noindent\begin{tabularx}{\linewidth}{XXXX}
		\toprule
		\textbf{Command}                     & \textbf{location}   & \textbf{Description}                                               & \textbf{Example}          \\
		% ————————————————————————————————————————————————————————————
		\midrule
		\textbf{commands/editor}                                                                                                                                    \\
		\midrule

		\mintinline{latex}{\citationrequise} & \colorize{main.tex} & Avertissement pour l'éditeur : une citation est à insérer ici      & \citationrequise          \\ \\

		\mintinline{latex}{\exemplerequis}   & \colorize{main.tex} & Avertissement pour l'éditeur : un exemple est à insérer ici        & \exemplerequis            \\ \\

		\mintinline{latex}{\editorwarn}      & \colorize{main.tex} & Avertissement pour l'éditeur                                       & \editorwarn{texte custom} \\ \\

		\mintinline{latex}{\editlater}       & \colorize{main.tex} & Avertissement pour l'éditeur : une modification est à apporter ici & \editlater{texte custom}  \\ \\
		\bottomrule
		% ————————————————————————————————————————————————————————————
	\end{tabularx}

	\pagebreak

	\noindent\begin{tabularx}{\linewidth}{X}
		\toprule
		\textbf{commands/graphics/\faAsterisk}                                                                            \\
		\midrule
		\textbf{Description}                                                                                              \\
		Displays an environment delimited with a blue line on the left, with an Info Icon located at the left of the line \\
		\midrule
	\end{tabularx}
	\noindent\begin{tabularx}{\linewidth}{XXXX}
		\textbf{Command}              & \textbf{location}         & \textbf{color}                                                                                     & \textbf{symbol}                                                                \\
		\midrule

		\mintinline{latex}{\info}     & \colorize{awesomebox.tex} & \colorize[flatuicolors_blue]{\detokenize{flatuicolors_blue}}                                       & \smallskip\textbf{symbol :} \colorize[flatuicolors_blue]{\faInfoCircle}        \\ \\

		\mintinline{latex}{\chk}      & \colorize{awesomebox.tex} & \colorize[flatuicolors_green]{\detokenize{flatuicolors_green}}                                     & \textbf{symbol :} \colorize[flatuicolors_green]{\faCheckCircle}                \\ \\

		\mintinline{latex}{\brain}    & \colorize{awesomebox.tex} & \colorize[flatuicolors_purple_light]{\detokenize{flatuicolors_purple_}\linebreak\smallskip light}  & \textbf{symbol :} \colorize[flatuicolors_purple_light]{\faBrain}               \\ \\

		\mintinline{latex}{\warn}     & \colorize{awesomebox.tex} & \colorize[flatuicolors_orange_light]{\detokenize{flatuicolors_orange_}\linebreak\smallskip light}  & \textbf{symbol :} \colorize[flatuicolors_orange_light]{\faExclamationTriangle} \\ \\

		\mintinline{latex}{\nope}     & \colorize{awesomebox.tex} & \colorize[flatuicolors_red_light]{\detokenize{flatuicolors_red_light}}                             & \textbf{symbol :} \colorize[flatuicolors_red_light]{\faTimesCircle}            \\ \\

		\mintinline{latex}{\cogs}     & \colorize{awesomebox.tex} & \colorize[flatuicolors_imperial]{\detokenize{flatuicolors_imperial}}                               & \textbf{symbol :} \colorize[flatuicolors_imperial]{\faCogs}                    \\ \\

		\mintinline{latex}{\citer}    & \colorize{awesomebox.tex} & \colorize[flatuicolors_corn_flower]{\detokenize{flatuicolors_corn_}\linebreak\smallskip flower}    & \textbf{symbol :} \colorize[flatuicolors_corn_flower]{\faQuoteRight}           \\ \\

		\mintinline{latex}{\avion}    & \colorize{awesomebox.tex} & \colorize[flatuicolors_purple_dark]{\detokenize{flatuicolors_purple_}\linebreak\smallskip dark}    & \textbf{symbol :} \colorize[flatuicolors_purple_dark]{\faFighterJet}           \\ \\

		\mintinline{latex}{\question} & \colorize{awesomebox.tex} & \colorize[flatuicolors_aqua] {\detokenize{flatuicolors_aqua}}                                      & \textbf{symbol :} \colorize[flatuicolors_aqua]{\faQuestionCircle}              \\ \\

		\mintinline{latex}{\idee}     & \colorize{awesomebox.tex} & \colorize[flatuicolors_yellow] {\detokenize{flatuicolors_yellow}}                                  & \textbf{symbol :} \colorize[flatuicolors_yellow]{\faLightbulb}                 \\ \\

		\mintinline{latex}{\book}     & \colorize{awesomebox.tex} & \colorize[flatuicolors_orange_light] {\detokenize{flatuicolors_orange_}\linebreak\smallskip light} & \textbf{symbol :} \colorize[flatuicolors_orange_light]{\faBook}                \\ \\

		\mintinline{latex}{\flask}    & \colorize{awesomebox.tex} & \colorize[flatuicolors_blue_devil]{\detokenize{flatuicolors_blue_}\linebreak\smallskip devil}      & \textbf{symbol :} \colorize[flatuicolors_blue_devil]{\faFlask}                 \\

		% ————————————————————————————————————————————————————————————
		\bottomrule
	\end{tabularx}
	\pagebreak



	\noindent\begin{tabularx}{\linewidth}{X}
		\toprule
		\textbf{commands/graphics/\faAsterisk}                                                                            \\
		\midrule
		\textbf{Description}                                                                                              \\
		Displays an environment delimited with a blue line on the left, with an Info Icon located at the left of the line \\
		\midrule
	\end{tabularx}
	\noindent\begin{tabularx}{\linewidth}{XXXX}
		\textbf{Command}                 & \textbf{location} & \textbf{short desc.}       & \textbf{Example}                           \\
		\midrule
		\\
		\mintinline{latex}{\blackboxed}  & blackbox.tex      & black rect. box            & \blackboxed{custom text}                   \\ \\

		\mintinline{latex}{\greenboxed}  & blackbox.tex      & green rect. box            & \greenboxed{custom text}                   \\ \\

		\mintinline{latex}{\blueboxed}   & blackbox.tex      & blue rect. box             & \blueboxed{custom text}                    \\ \\

		\mintinline{latex}{\purpleboxed} & blackbox.tex      & purple rect. box           & \purpleboxed{custom text}                  \\ \\

		\mintinline{latex}{\orangeboxed} & blackbox.tex      & orange rect. box           & \orangeboxed{custom text}                  \\ \\

		\mintinline{latex}{\redboxed}    & blackbox.tex      & red rect. box              & \redboxed{custom text}                     \\ \\

		\mintinline{latex}{\aquaboxed}   & blackbox.tex      & aqua rect. box             & \aquaboxed{custom text}                    \\ \\

		\mintinline{latex}{\icon}        & blackbox.tex      & fontawesome icon with text & \icon{Github}{10}{GitHub}                  \\ \\

		\midrule

		\mintinline{latex}{\circled}     & circled.tex       & circled text               & \circled{1}                                \\ \\

		\midrule

		\mintinline{latex}{\colorize}    & colorize.tex      & colored text               & \colorize[flatuicolors_green]{custom text} \\ \\

		\bottomrule
	\end{tabularx}
	\pagebreak

	\noindent\begin{tabularx}{\linewidth}{X}
		\toprule
		\textbf{commands/maths/\faAsterisk}                                                               \\
		\midrule
		\textbf{Description}                                                                              \\
		The commands associated with symbols and other things for mathematics / mathematical environments \\
		\midrule
	\end{tabularx}
	\noindent\begin{tabularx}{\linewidth}{XXXX}
		\textbf{Command}                                                                   & \textbf{location}              & \textbf{short desc.}                               & \textbf{Example}                                 \\
		\midrule
		\mintinline{latex}{\P}                                                             & \detokenize{proba_lettres.tex} & Probabilité                                        & $\P$                                            \\ \\
		\mintinline{latex}{\E}                                                             & \detokenize{proba_lettres.tex} & Espérance                                          & $\E$                                             \\ \\
		\mintinline{latex}{\V}                                                             & \detokenize{proba_lettres.tex} & Variance                                           & $\V$                                             \\ \\
		\mintinline{latex}{\Q}                                                             & \detokenize{proba_lettres.tex} & Rationels                                          & $\Q$                                             \\ \\
		\mintinline{latex}{\IR}                                                            & \detokenize{proba_lettres.tex} & Réels                                              & $\IR$                                            \\ \\
		\mintinline{latex}{\IH}                                                            & \detokenize{proba_lettres.tex} & Hilbert                                            & $\IH$                                            \\ \\
		\midrule                                                                                                                                                                                                                    \\
		\mintinline{latex}{\indep}                                                         & \detokenize{proba.tex}         & symbole indép                                      & $\indep$                                         \\ \\
		\hyperref[code:samelaw]{\mintinline{latex}{\samelaw}\label{desc:samelaw} }         & \detokenize{proba.tex}         & suit la loi de                                     & $X \samelaw Z/\sigma$                            \\ \\
		\mintinline{latex}{\proba}                                                         & \detokenize{proba.tex}         & Probabilité de                                     & $\proba{\lvert X \rvert > \varepsilon}$          \\ \\
		\hyperref[code:probaloi]{\mintinline{latex}{\probaloi}\label{desc:probaloi}}       & \detokenize{proba.tex}         & Probabilité de $[\cdot]$ selon la loi de $[\cdot]$ & $\probaloi{X | Y}{2X^2 - 7Y < \eta}$             \\ \\
		\mintinline{latex}{\variance}                                                      & \detokenize{proba.tex}         & Variance de $[\cdot]$                              & $\variance{\widehat X}$                          \\ \\
		\mintinline{latex}{\esperance}                                                     & \detokenize{proba.tex}         & Espérance de $[\cdot]$                             & $\esperance{\widehat \theta}$                    \\ \\
		\mintinline{latex}{\esperanceloi}                                                  & \detokenize{proba.tex}         & Espérance de $[\cdot]$ selon la loi de $[\cdot]$   & $\esperanceloi{Y | X}{Y - X}$                    \\ \\
		\mintinline{latex}{\esperancesachant}                                              & \detokenize{proba.tex}         & Espérance conditionnelle                           & $\esperancesachant{X}{Y}$                        \\ \\
		\hyperref[code:esploisach]{\mintinline{latex}{\esploisach}}\label{desc:esploisach} & \detokenize{proba.tex}         & Espérance conditionnelle selon une loi             & $\esploisach{Z}{ ZU \times \log(\sigma)Z ^2}{U}$ \\ \\
		\midrule                                                                                                                                                                                                                    \\
		\mintinline{latex}{\orthonorm}                                                     & \detokenize{property.tex}      & symbol orthonormal                                 & $u \, \orthonorm \, \calF$                       \\ \\
		\midrule
		% \mintinline{latex}{} & \detokenize{proba.tex} & & $$ \\ \\
		% \mintinline{latex}{} & \detokenize{proba.tex} & & $$ \\ \\
	\end{tabularx}


	\noindent\begin{tabularx}{\linewidth}{XXXX}
		\textbf{Command}                           & \textbf{location} & \textbf{short desc.}                                                          & \textbf{Example}                            \\
		\midrule
		% ————————————————————————————————————————————————————————————
		\mintinline{latex}{\cvl}                   & convergence.tex   & convergence en loi                                                            & $u_n \cvl{n}{+\infty} \ell$                 \\ \\

		\mintinline{latex}{\cvp}                   & convergence.tex   & convergence en probabilité                                                    & $u_n \cvp{n}{+\infty} \ell$                 \\ \\

		\mintinline{latex}{\cvps}                  & convergence.tex   & convergence presque sûre                                                      & $u_n \cvps{n}{+\infty} \ell$                \\ \\

		\mintinline{latex}{\cvL}                   & convergence.tex   & convergence $\mathds L^p$                                                     & $u_n \cvL{p}{n}{+\infty} \ell$              \\ \\

		\mintinline{latex}{\cvetr}                 & convergence.tex   & convergence étroite                                                           & $u_n \cvetr{n}{+\infty} \ell$               \\ \\

		\mintinline{latex}{\cvnorme}               & convergence.tex   & convergence en norme                                                          & $u_n \cvnorme{n}{+\infty} \ell$             \\ \\

		\mintinline{latex}{\cvpp}                  & convergence.tex   & convergence presque partout                                                   & $u_n \cvpp{n}{+\infty} \ell$                \\ \\

		\mintinline{latex}{\tendset}               & convergence.tex   & tend vers dans un ensemble                                                    & $u_n \tendset{n}{+\infty}{\mathcal F} \ell$ \\ \\

		\midrule

		\mintinline{latex}{\intervaleint}          & ensembles.tex     & intervalle entier                                                             & $\intervaleint{p}{q}$                       \\ \\

		\mintinline{latex}{\R}                     & ensembles.tex     & espace $\mathds R^p$                                                          & $\R{p}$                                     \\ \\

		\mintinline{latex}{\classespace}           & ensembles.tex     & espace des fonctions de classe $k$ sur un ensemble $E$                        & $\classespace{k}{E}$                        \\ \\

		\mintinline{latex}{\continuborne}          & ensembles.tex     & espace des fonctions continues et bornées sur un ensemble $E$ dans $F$        & $\continuborne{E}{F}$                       \\ \\

		\mintinline{latex}{\continusupportcompact} &                   & espace des fonctions continues à support compact sur un ensemble $E$ dans $F$ & $\continusupportcompact{E}{F}$              \\ \\
	\end{tabularx}


	\noindent\begin{tabularx}{\linewidth}{XXXX}
		\mintinline{latex}{\mesurable}                                                                      & ensembles.tex & espace des fonctions mesurables sur un ensemble $E$ dans $F$               & $\mesurable{E}{F}$                            \\ \\

		\mintinline{latex}{\etageepositive}                                                                 & ensembles.tex & espace des fonctions etagées positives sur un ensemble $E$ dans $F$        & $\etageepositive{E}{F}$                       \\ \\

		\mintinline{latex}{\VA}                                                                             & ensembles.tex & espace des variables aléatoires à valeur dans $E$                          & $\VA{E}$                                      \\ \\

		\mintinline{latex}{\matrixspace}                                                                    & ensembles.tex & espace des matrices carrées de taille $p \times p$ à coefficients dans $E$ & $\matrixspace{p}{E}$                          \\ \\

		\mintinline{latex}{\orthonormal}                                                                    & ensembles.tex & symbole orthonormal                                                        & $\orthonormal$                                \\ \\

		\mintinline{latex}{\orthonormalselon}                                                               & ensembles.tex & symbole orthonormal selon un produit scalaire                              & $\orthonormalselon{\mathds L^2}$              \\ \\

		\midrule

		\mintinline{latex}{\grandR}                                                                         & ensembles.tex & symbole de l'ensemble des réels                                            & $\grandR$                                     \\ \\


		H / T / J / W / F / X / Y / F / I / E / M / B / N / Z / Q / C / K                                   &               & autres lettres disponibles                                                                                                 \\ \\

		\mintinline{latex}{\calR}                                                                           & ensembles.tex & symbole de l'ensemble des entiers naturels                                 & $\calR$                                       \\ \\

		F / O / L / P / M / N / A / B / C / D / E / F / G / H / I / J / K / Q                               &               & autres lettres disponibles                                                                                                 \\ \\

		\mintinline{latex}{\Rplus} / \mintinline{latex}{\Rmoins}                                            & ensembles.tex & symbole de l'ensemble des réels positifs / négatifs                        & $\Rplus$ / $\Rmoins$                          \\ \\

		\mintinline{latex}{\Rplusetoile} / \mintinline{latex}{\Rmoinsetoile} / \mintinline{latex}{\Retoile} & ensembles.tex & symbole de l'ensemble des réels positifs / négatifs non nuls               & $\Rplusetoile$ / $\Rmoinsetoile$ / $\Retoile$ \\ \\
	\end{tabularx}
	% \pagebreak

	\noindent\begin{tabularx}{\linewidth}{XXXX}
		\textbf{Command}                     & \textbf{location}                                               & \textbf{short desc.}                            & \textbf{Example}               \\
		\midrule
		% ————————————————————————————————————————————————————————————
		\mintinline{latex}{\indicatrice}     & \detokenize{fonctions_et_}\linebreak\detokenize{operateurs.tex} & indicatrice d'un ensemble                       & $\indicatrice{A}$              \\ \\

		\mintinline{latex}{\norme}           & \detokenize{fonctions_et_}\linebreak\detokenize{operateurs.tex} & norme d'un élément                              & $\norme{p}{x}$                 \\ \\

		\mintinline{latex}{\dist}            & \detokenize{fonctions_et_}\linebreak\detokenize{operateurs.tex} & distance issue d'une norme entre deux vecteurs  & $\dist{x}{y}$                  \\ \\

		\mintinline{latex}{\distnorme}       & \detokenize{fonctions_et_}\linebreak\detokenize{operateurs.tex} & distance issue d'une norme entre deux vecteurs  & $\distnorme{\infty}{x}{y}$     \\ \\

		\mintinline{latex}{\prodscal}        & \detokenize{fonctions_et_}\linebreak\detokenize{operateurs.tex} & produit scalaire entre deux vecteurs            & $\prodscal{x}{y}$              \\ \\

		\mintinline{latex}{\prodscalselon}   & \detokenize{fonctions_et_}\linebreak\detokenize{operateurs.tex} & produit scalaire [spécifié] entre deux vecteurs & $\prodscalselon{x}{y}{\infty}$ \\ \\

		\mintinline{latex}{\argmax(\limits)} & \detokenize{fonctions_et_}

		operateurs.tex                       & argmax                                                          & $\argmax\limits_{x \in E} f(x)$                                                  \\ \\

		\mintinline{latex}{\argmin(\limits)} & \detokenize{fonctions_et_}

		operateurs.tex                       & argmin                                                          & $\argmin\limits_{x \in E} f(x)$                                                  \\ \\

		\mintinline{latex}{\inverse}         & \detokenize{fonctions_et_}

		operateurs.tex                       & inverse d'un élément                                            & $\inverse{A}$                                                                    \\ \\

		\mintinline{latex}{\isdef}           & \detokenize{fonctions_et_}

		operateurs.tex                       & est défini comme                                                & $A \isdef B$                                                                     \\ \\

		\mintinline{latex}{\comm}            & \detokenize{fonctions_et_}

		operateurs.tex                       & commutant d'un ensemble d'opérateurs                            & $\comm{A}$                                                                       \\ \\

		\mintinline{latex}{\rg}              & \detokenize{fonctions_et_}

		operateurs.tex                       & rang d'un élément                                               & $\rg{A}$                                                                         \\ \\

		\mintinline{latex}{\im}              & \detokenize{fonctions_et_}

		operateurs.tex                       & image d'un élément                                              & $\im{A}$                                                                         \\ \\

		\mintinline{latex}{\pgcd}            & \detokenize{fonctions_et_}

		operateurs.tex                       & pgcd                                                            & $\pgcd{p}{q}$                                                                    \\ \\

		\mintinline{latex}{\positive}        & \detokenize{fonctions_et_}

		operateurs.tex                       & partie positive d'un élément                                    & $\positive{x^3 - x^2}$                                                           \\ \\
	\end{tabularx}

	\noindent\begin{tabularx}{\linewidth}{XXXX}
		\hyperref[code:func]{\mintinline{latex}{\func}} & \detokenize{fonctions_et_}\linebreak\detokenize{operateurs.tex} & définition d'une fonction                            & $f: \func{E}{F}{x}{f(x)}$   \\\\

		\midrule                                                                                                                                                                                               \\

		\mintinline{latex}{\petitop}                    & \detokenize{limites.tex}                                        & petit o en probabilité                               & $\petitop{n^{- \frac 1 5}}$ \\ \\

		\mintinline{latex}{\grandop}                    & \detokenize{limites.tex}                                        & grand O en probabilité                               & $\grandop{n^{- \frac 1 5}}$ \\ \\

		\midrule                                                                                                                                                                                               \\

		\mintinline{latex}{\statrang}                   & \detokenize{suites.tex}                                         & $k^e$ valeur ordonnée (ordre croissant)              & $\statrang Y n k$           \\ \\
		\mintinline{latex}{\suiteensemble}              & \detokenize{suites.tex}                                         & suite à valeur dans $E$                              & $\suiteensemble E$          \\ \\
		\mintinline{latex}{\suite}                      & \detokenize{suites.tex}                                         & suite \og u n \fg                                    & $\suite u n$                \\ \\
		\mintinline{latex}{\soussuite}                  & \detokenize{suites.tex}                                         & sous suite indexée par $k$                           & $\soussuite u k$            \\ \\
		\mintinline{latex}{\famille}                    & \detokenize{suites.tex}                                         & famille d'objets indexée sur un ensemble $I$         & $\famille {\mathds X}{i}$   \\ \\
		\mintinline{latex}{\suitecomposition}           & \detokenize{suites.tex}                                         & suite d'images d'une suite $x_k$ par la fonction $f$ & $\suitecomposition f x k$   \\ \\
		% TODO : trouver à quoi ça correspond (stat maths)
		\mintinline{latex}{\suitestatrang}              & \detokenize{suites.tex}                                         & ???                                                  & $\suitestatrang X \eta k$   \\ \\
		\mintinline{latex}{\famfinie}                   & \detokenize{suites.tex}                                         & ensemble fini d'éléments de $[\cdot]$ à $[\cdot]$    & $\famfinie x 1 n$           \\ \\
		\mintinline{latex}{\fromto}                     & \detokenize{suites.tex}                                         & de $[\cdot]$ à $[\cdot]$                             & $\fromto X 1 p$             \\ \\
		\mintinline{latex}{\ordered}                    & \detokenize{suites.tex}                                         & élément ordonné (ici $k^e$)                          & $\ordered X k$              \\ \\
		\bottomrule
	\end{tabularx}
	

	\noindent\begin{tabularx}{\linewidth}{XXXX}
		\toprule                                                                                            \\
		% Lebesgue
		\mintinline{latex}{\leb} & \detokenize{integral.tex} & Intégrale de Lebesgue (symbol différenciel) & $\leb$ \\ \\
		\mintinline{latex}{\lebesgue} & \detokenize{integral.tex} & Intégrale de Lebesgue $\oplus$ ensemble   & $\lebesgue {\mathds X}$ \\ \\
		\mintinline{latex}{\lebint} & \detokenize{integral.tex} & Intégrale de Lebesgue $\oplus$ de $a$ à $b$  & $\lebint a b$ \\ \\
		\mintinline{latex}{\lebm} & \detokenize{integral.tex} & Intégrale de Lebesgue (ensemble $\oplus$ intégrande $\oplus$ mesure )  & $\lebm {\mathds X} f \mu$ \\ \\
		% Bochner
		\midrule
		\mintinline{latex}{\boch} & \detokenize{integral.tex} & Intégrale de Bochner (symbol différenciel) & $\boch$ \\ \\
		\mintinline{latex}{\bochner} & \detokenize{integral.tex} & Intégrale de Bochner $\oplus$ ensemble   & $\bochner {\mathds X}$ \\ \\
		\mintinline{latex}{\bochint} & \detokenize{integral.tex} & Intégrale de Bochner $\oplus$ de $a$ à $b$  & $\bochint a b$ \\ \\
		\mintinline{latex}{\bochm} & \detokenize{integral.tex} & Intégrale de Bochner (ensemble $\oplus$ intégrande $\oplus$ mesure )  & $\bochm {\mathds X} f \mu$ \\ \\
		\midrule
	\end{tabularx}
	\noindent\begin{tabularx}{\linewidth}{XXXX}
		\midrule \\
		% Riemann
		\mintinline{latex}{\riem} & \detokenize{integral.tex} & Intégrale de Riemann (symbol différenciel) & $\riem$ \\ \\
		\mintinline{latex}{\riemann} & \detokenize{integral.tex} & Intégrale de Riemann $\oplus$ ensemble   & $\riemann {\mathds X}$ \\ \\
		\mintinline{latex}{\riemint} & \detokenize{integral.tex} & Intégrale de Riemann $\oplus$ de $a$ à $b$  & $\riemint a b$ \\ \\
		\mintinline{latex}{\riemm} & \detokenize{integral.tex} & Intégrale de Riemann (ensemble $\oplus$ intégrande $\oplus$ mesure )  & $\riemm {\mathds X} f \mu$ \\ \\
		% Pettis
		\midrule
		\mintinline{latex}{\pet} & \detokenize{integral.tex} & Intégrale de Pettis (symbol différenciel) & $\pet$ \\ \\
		\mintinline{latex}{\pettis} & \detokenize{integral.tex} & Intégrale de Pettis $\oplus$ ensemble   & $\pettis {\mathds X}$ \\ \\
		\mintinline{latex}{\petint} & \detokenize{integral.tex} & Intégrale de Pettis $\oplus$ de $a$ à $b$  & $\petint a b$ \\ \\
		\mintinline{latex}{\petm} & \detokenize{integral.tex} & Intégrale de Pettis (ensemble $\oplus$ intégrande $\oplus$ mesure )  & $\petm {\mathds X} f \mu$ \\ \\
		\bottomrule
	\end{tabularx}

	\pagebreak
	\noindent\begin{tabularx}{\linewidth}{X}
		\toprule
		\textbf{definition/custom\_colors.tex}                                                              \\
		\midrule
		\textbf{Description}                                                                                \\
		Custom colors that can be used in other commands such as \mintinline{latex}{\colorize[color]{text}} \\
		\midrule
	\end{tabularx}
	\noindent\begin{tabularx}{\linewidth}{XX}
		\textbf{color name}                    & \textbf{color}                                            \\
		\midrule
		\detokenize{flatuicolors_orange}       & \colorbox{flatuicolors_orange}{ \, \, \, \, \, \, }       \\ \\

		\detokenize{flatuicolors_orange_light} & \colorbox{flatuicolors_orange_light}{ \, \, \, \, \, \, } \\ \\

		\detokenize{flatuicolors_red_light}    & \colorbox{flatuicolors_red_light}{ \, \, \, \, \, \, }    \\ \\

		\detokenize{flatuicolors_tomato}       & \colorbox{flatuicolors_tomato}{ \, \, \, \, \, \, }       \\ \\

		\detokenize{flatuicolors_yellow}       & \colorbox{flatuicolors_yellow}{ \, \, \, \, \, \, }       \\ \\

		\detokenize{flatuicolors_green}        & \colorbox{flatuicolors_green}{ \, \, \, \, \, \, }        \\ \\

		\detokenize{flatuicolors_greenish}     & \colorbox{flatuicolors_greenish}{ \, \, \, \, \, \, }     \\ \\

		\detokenize{flatuicolors_blue}         & \colorbox{flatuicolors_blue}{ \, \, \, \, \, \, }         \\ \\

		\detokenize{flatuicolors_blue_light}   & \colorbox{flatuicolors_blue_light}{ \, \, \, \, \, \, }   \\ \\

		\detokenize{flatuicolors_blue_deep}    & \colorbox{flatuicolors_blue_deep}{ \, \, \, \, \, \, }    \\ \\

		\detokenize{flatuicolors_blue_devil}   & \colorbox{flatuicolors_blue_devil}{ \, \, \, \, \, \, }   \\ \\

		\detokenize{flatuicolors_purple}       & \colorbox{flatuicolors_purple}{ \, \, \, \, \, \, }       \\ \\

		\detokenize{flatuicolors_purple_light} & \colorbox{flatuicolors_purple_light}{ \, \, \, \, \, \, } \\ \\

		\detokenize{flatuicolors_purple_dark}  & \colorbox{flatuicolors_purple_dark}{ \, \, \, \, \, \, }  \\ \\

		\detokenize{flatuicolors_rose}         & \colorbox{flatuicolors_rose}{ \, \, \, \, \, \, }         \\ \\

		\detokenize{flatuicolors_biscay}       & \colorbox{flatuicolors_biscay}{ \, \, \, \, \, \, }       \\ \\

		\detokenize{flatuicolors_imperial}     & \colorbox{flatuicolors_imperial}{ \, \, \, \, \, \, }     \\ \\

		\detokenize{flatuicolors_aqua}         & \colorbox{flatuicolors_aqua}{ \, \, \, \, \, \, }         \\ \\

		\detokenize{flatuicolors_magenta}      & \colorbox{flatuicolors_magenta}{ \, \, \, \, \, \, }      \\ \\

		\detokenize{flatuicolors_light_gray}   & \colorbox{flatuicolors_light_gray}{ \, \, \, \, \, \, }   \\ \\


		\bottomrule
	\end{tabularx}

	\pagebreak

	\subsection{Commands Code Examples}

	\noindent\begin{tabularx}{\linewidth}{XXXX}
		\toprule
		\textbf{Command}                                                                   & \textbf{Arguments}                                                                      & \textbf{Code}                                                                        & \textbf{Render} \\
		\midrule
		% ~ func ——————————————————————————
		\mintinline{latex}{\func}\label{code:func}                                         & \begin{enumerate}
																								\item \mintinline{latex}{{E}}
																								\item \mintinline{latex}{{F}}
																								\item \mintinline{latex}{{x}}
																								\item \mintinline{latex}{{f(x)}}
																							\end{enumerate}                       & \mintinline{latex}{f: \func{E}{F}}\linebreak\mintinline{latex}{{x}{f(x)}}            & $f: \func{E}{F}{x}{f(x)}$                                         \\ \\
		% ~————————————————————————————————
		\midrule                                                                                                                                                                                                                                                                              \\
		% ~ samelaw ———————————————————————
		\hyperref[desc:samelaw]{\mintinline{latex}{\samelaw}}\label{code:samelaw}          & \begin{enumerate}
																								\item \textbf{loi suivie :} \mintinline{latex}{{Z}}
																							\end{enumerate} & \mintinline{latex}{X \samelaw Z}                                                     & $X \samelaw Z$                                                                          \\ \\
		% ~————————————————————————————————
		\midrule                                                                                                                                                                                                                                                                              \\
		% ~ samelaw ———————————————————————
		\hyperref[desc:probaloi]{\mintinline{latex}{\probaloi}}\label{code:probaloi}       & \begin{enumerate}
																								\item \textbf{loi :} \mintinline{latex}{{X}}
																								\item \textbf{expression :} \mintinline{latex}{{X^2}}
																							\end{enumerate} & \mintinline{latex}{\probaloi{X | Y}}\linebreak\mintinline{latex}{{2X^2 - 7Y < \eta}} & $\probaloi{X | Y}{2X^2 - 7Y < \eta}$                                                    \\ \\
		% ~————————————————————————————————
		\midrule                                                                                                                                                                                                                                                                              \\
		% ~ esploisach ————————————————————
		\hyperref[desc:esploisach]{\mintinline{latex}{\esploisach}}\label{code:esploisach} & \begin{enumerate}
																								\item \textbf{loi :} \mintinline{latex}{{Z}}
																								\item \textbf{expression :} \mintinline{latex}{{Z \times \log U}}
																								\item \textbf{sachant :} \mintinline{latex}{{U}}
																							\end{enumerate} &
		\mintinline{latex}{\esploisach{Z}}
		\mintinline{latex}{{Z \times \log U}}
		\mintinline{latex}{{U}}
																						& $\esploisach{Z}{ ZU \times \log(\sigma)Z ^2}{U}$                                                                                                                                                 \\ \\
		\bottomrule
	\end{tabularx}

\end{document}