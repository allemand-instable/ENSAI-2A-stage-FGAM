\newcommand{\custint}[1]{{}_{#1}\displaystyle\int}

% Lebesgue
\newcommand{\leb}{ \custint{\mathcal{L}} }
\newcommand{\lebm}[3]{ \custint{\mathcal{L}}_{#1} \, #2 \ d{#3} }
\newcommand{\lebesgue}[1]{ \custint{\mathcal{L}}_{#1} }
\newcommand{\lebint}[2]{ \custint{\mathcal{L}}_{#1}^{#2} }

% Dunford
\newcommand{\dun}{ \custint{\mathds{D}} }
\newcommand{\dunm}[3]{ \custint{\mathds{D}}_{#1} \, #2 \ d{#3} }
\newcommand{\dunford}[1]{ \custint{\mathds{D}}_{#1} }
\newcommand{\dunint}[2]{ \custint{\mathds{D}}_{#1}^{#2} }

% Bochner
\newcommand{\boch}{ \custint{\mathds{B}} }
\newcommand{\bochm}[3]{ \custint{\mathds{B}}_{#1} \, #2 \ d{#3} }
\newcommand{\bochner}[1]{ \custint{\mathds{B}}_{#1} }
\newcommand{\bochint}[2]{ \custint{\mathds{B}}_{#1}^{#2} }

% Riemann
\newcommand{\riem}{ \custint{\mathcal{R}} }
\newcommand{\riemm}[3]{ \custint{\mathcal{R}}_{#1} \, #2 \ d{#3} }
\newcommand{\riemann}[1]{ \custint{\mathcal{R}}_{#1} }
\newcommand{\riemint}[2]{ \custint{\mathcal{R}}_{#1}^{#2} }

% Pettis-Gelfand
\newcommand{\pet}{ \custint{\mathcal{P}} }
\newcommand{\petm}[3]{ \custint{\mathcal{P}}_{#1} \, #2 \ d{#3} }
\newcommand{\pettis}[1]{ \custint{\mathcal{P}}_{#1} }
\newcommand{\petint}[2]{ \custint{\mathcal{P}}_{#1}^{#2} }
