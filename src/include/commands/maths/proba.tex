\newcommand{\indep}{\perp \!\!\! \perp}

\newcommand{\samelaw}{\overset{\mathcal{L}}{\, \sim \,}}

%   fonctions mathématiques de proba


% & Name
%       \proba
% & Description
%       Probabilité d'un évènement : ℙ[ ⋯ ]
% & Arguments
%       1 : Evenement
% & Examples
%       \proba{ X \in \mathcal A }
\newcommand{\proba}[1]{\mathds{P} \left[ \, #1 \, \right]}
% & Name
%       \probaloi
% & Description
%       Probabilité d'un évènement suivant la loi spécifiée : ℙₓ[ ⋯ ]
% & Arguments
%       1 : loi
%       2 : Evenement
% & Examples
%       \probaloi{X \cap Y}{ XY = 7 }
\newcommand{\probaloi}[2]{\mathds{P}_{#1} \left[ \, #2 \, \right]}
% & Name
%       \variance
% & Description
%       variance d'une variable aléatoire : 𝕍[ ⋯ ]
% & Arguments
%       1 : Variable Aléatoire
% & Examples
%       \variance{ \hat\theta }
\newcommand{\variance}[1]{\mathds{V} \left[ \, #1 \, \right]}
% & Name
%       \esperance
% & Description
%       Espérance d'une variable aléatoire
% & Arguments
%       
% & Examples
%       
\newcommand{\esperance}[1]{\mathds{E} \left[ \, #1 \, \right]}
% & Name
%       \esperanceloi
% & Description
%       Espérance d'une variable aléatoire selon la loi spécifiée : 𝔼ₓ[ ⋯ ]
% & Arguments
%       1 : loi
%       2 : variable aléatoire
% & Examples
%       
\newcommand{\esperanceloi}[2]{\mathds{E}_{#1} \left[ \, #2 \, \right]}
% & Name
%       \esperancesachant
% & Description
%       Espérance conditionnelle 𝔼[ Y | X ]
% & Arguments
%       1 : conditionnement
%       2 : évènement
% & Examples
%       \esperancesachant{X}{Y}
\newcommand{\esperancesachant}[2]{\mathds{E} \left[ \, #2 \, \vert \, #1 \, \right]}
% & Name
%       \esploisach
% & Description
%       Espérance conditionnelle selon la loi spécifiée : 𝔼ₚ[ Y | X ]
% & Arguments
%       1 : loi
%       2 : conditionnement
%       3 : evenement
% & Examples
%       \esploisach{p}{X}{Y}
\newcommand{\esploisach}[3]{{\mathds{E}}_{#1}\left[ \, #3 \, \vert \, #2 \, \right]}
