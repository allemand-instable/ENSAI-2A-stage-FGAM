%%  notation convergence
% uses mathtools package
% & Name
% 		\cvl
% & Description
% 		Flèche de la convergence en loi
% & Arguments
% 		1: variable de la convergence
% 		2: target de la convergence
% & Examples
% 		\cvl{n}{+\infty}
\newcommand{\cvl}[2]{\xrightarrow[#1 \rightarrow #2]{\ \mathcal L \ } \ }

% & Name
% 		\cvp
% & Description
% 		Flèche de la convergence en probabilité
% & Arguments
% 		1: variable de la convergence
% 		2: target de la convergence
% & Examples
% 		\cvp{n}{+\infty}
\newcommand{\cvp}[2]{\xrightarrow[#1 \rightarrow #2]{\ \mathbb P \ } }

% & Name
% 		\cvps
% & Description
% 		Flèche de la convergence presque sûre
% & Arguments
% 		1: variable de la convergence
% 		2: target de la convergence
% & Examples
% 		\cvps{n}{+\infty}
\newcommand{\cvps}[2]{\xrightarrow[#1 \rightarrow #2]{\ \textsf{p.s}\ } \ }

% & Name
% 		\cvL
% & Description
% 		Flèche de la convergence en L^p
% & Arguments
% 		1: p
% 		2: variable de la convergence
% 		3: target de la convergence
% & Examples
% 		\cvL{p}{n}{+\infty}
\newcommand{\cvL}[3]{\xrightarrow[#2 \rightarrow #3]{\ \mathbb L ^{#1} \ }}

% & Name
% 		\cvetr
% & Description
% 		Flèche de la convergence étroite
% & Arguments
% 		1: variable de la convergence
% 		2: target de la convergence
% & Examples
% 		\cvetr{n}{+\infty}
\newcommand{\cvetr}[2]{\xrightarrow[#1 \rightarrow #2]{ \ \textsf{étroit.} \ }}

% & Name
% 		\cvnorme
% & Description
% 		Flèche de la convergence en norme
% & Arguments
% 		1: subscript de la norme
% 		2: variable de la convergence
% 		3: target de la convergence
% & Examples
% 		\cvnorme{\infty}{x}{x_0}
\newcommand{\cvnorme}[3]{\xrightarrow[\, #2 \rightarrow #3 \; ]{\, \lVert \cdot \rVert _{#1} \,} \ }

% & Name
% 		\cvpp
% & Description
% 		Flèche de la convergence en probabilité [mesure] presque partout
% & Arguments
% 		1: mesure
% 		2: variable de la convergence
% 		3: target de la convergence
% & Examples
% 		\cvpp{\mu}{n}{+\infty}
\newcommand{\cvpp}[3]{ \xrightarrow[#2 \rightarrow #3]{#1 - p.p} }
%%  autres
% & Name
% 		\tend
% & Description
% 		Flèche de la convergence
% & Arguments
% 		1: variable de la convergence
% 		2: target de la convergence
% & Examples
% 		\tend{n}{+\infty}
\newcommand{\tend}[2]{\xrightarrow[ #1 \rightarrow #2 ]{}}

% & Name
% 		\tendset
% & Description
% 		Flèche de la convergence avec un ensemble / superscript
% & Arguments
% 		1: variable de la convergence
% 		2: target de la convergence
% 		3: ensemble / superscript
% & Examples
% 		\tendset{n}{+\infty}{\mathcal F}
% 		\tendset{n}{+\infty}{\textsf{Unif.} }
\newcommand{\tendset}[3]{\xrightarrow[ #1 \rightarrow #2 ]{#3}}
