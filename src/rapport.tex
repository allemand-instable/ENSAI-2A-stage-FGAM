\documentclass[11pt]{report}

% ⚠️ DO NOT TOUCH ————————————
% * include and settings
% ✏️ note : include should be used in document environment → use of input here
% & definition
\newcounter{code}
\newcounter{proof}
% & value
\setcounter{code}{0}
\setcounter{proof}{0}
% ~ if statements
\ifnum\value{code}=1
\fi
\ifnum\value{proof}=1
\fi
\input{content/appendix/A/main.tex}
\input{content/appendix/B/main.tex}
\input{content/appendix/A/main.tex}
\input{content/appendix/B/main.tex}
\input{content/appendix/A/main.tex}
\input{content/appendix/B/main.tex}
\input{content/appendix/A/main.tex}
\input{content/appendix/B/main.tex}
% ✅ only text
\newcommand{\auteur}{BRUNET Hugo}
\newcommand{\articletitle}{Modèles additifs avec covariables contaminées et généralisation du modèle additif aux données fonctionnelles}
\newcommand{\customdate}{Novembre 2023}
\newcommand{\projecttitle}{Stage 2A}
\newcommand{\projectdescription}{CREST}
% ⚠️ HREF package
\newcommand{\mail}{\href{mailto:hugo.brunet@eleve.ensai.fr}{hugo.brunet@eleve.ensai.fr}}
\newcommand{\github}{\href{https://github.com/allemand-instable/}{allemand-instable}}
\newcommand{\githubissues}{\href{https://github.com/allemand-instable/ENSAI-2A-stage-FGAM/issues}{ENSAI-2A-stage-FGAM/issues}}

\input{include/bibliography.tex}
% ~ MINTED

% \newcommand{\setcoderender}[1]{%
% 	\newcounter{coderenderbool}%
% 	\setcounter{coderenderbool}{#1}%
% }

% \newcommand{\ifcode}[1]{%
% 	\ifnum1=\value{coderenderbool}%
% 	{%
% 		{#1}%
% 	}%
% 	\else
% 	{%
% 		%
% 	}%
% }


\ifnum\value{code}=1
	\usepackage[cache=false]{minted}

	\usemintedstyle{one-dark}
\fi

% document
\usepackage{pdfpages}
% TODO : ajouter à latex-template
% ! notations —————————————————
% TODO : ajouter à latex-template
\newcommand{\dash}{\rule[1pt]{0.5cm}{0.02cm}}
% ~ Noyaux
\newcommand{\Kh}[1]{K_h\left( \, #1 \, \right)}
\newcommand{\Khs}[1]{K_h^{\star}\left( \, #1 \, \right)}
\newcommand{\Khsd}[2]{K_h^{\star}\left( \, #1, #2 \, \right)}
\newcommand{\Khdev}[1]{\frac 1 h K\left( \frac{#1}{h} \right)}
\newcommand{\Kdeconvo}[1]{K^{[\textsf{deconv}]}_h\left( \, #1 \, \right)}
% ~ convolution
\newcommand{\convo}[2]{ \bigl[\,  #1  *  #2 \, \bigr] }
% ~ observations bruitées
\newcommand{\xtei}{\widetilde X^{*[\,i\,]}}
\newcommand{\yti}{\widetilde Y^{[\,i\,]}}
\newcommand{\zei}{Z^{*[\,i\,]}}
\newcommand{\xei}{X^{*[\,i\,]}}
\newcommand{\obsi}[1]{{#1}^{[\,i\,]}}
\newcommand{\yi}{\obsi{Y}}
\newcommand{\zi}{\obsi{Z}}
\newcommand{\vi}{\obsi{V}}
\newcommand{\ui}{\obsi{U}}

% ! —————————————————————————
% ✅ only text
\newcommand{\auteur}{BRUNET Hugo}
\newcommand{\articletitle}{Modèles additifs avec covariables contaminées et généralisation du modèle additif aux données fonctionnelles}
\newcommand{\customdate}{Novembre 2023}
\newcommand{\projecttitle}{Stage 2A}
\newcommand{\projectdescription}{CREST}
% ⚠️ HREF package
\newcommand{\mail}{\href{mailto:hugo.brunet@eleve.ensai.fr}{hugo.brunet@eleve.ensai.fr}}
\newcommand{\github}{\href{https://github.com/allemand-instable/}{allemand-instable}}
\newcommand{\githubissues}{\href{https://github.com/allemand-instable/ENSAI-2A-stage-FGAM/issues}{ENSAI-2A-stage-FGAM/issues}}

% ⚠️ —————————————————————————


% ⚠️ USE WHEN NO CITATION FROM BIB
\nocite{*}
% ⚠️ —————————————————————————


% ⚙️ Table of Contents
\dominitoc

% * global commands
% ⚙️ INCLUDE COMMAND
\input{include/global_commands.tex}

\begin{document}
% ~ PRE CONTENT  ——————————————————————
% > no page numbering
% ⧐ couverture  [settings]
% ⧐ abstract    [settings / abstract_text.tex]
% ⧐ notations   [content/pre/main.tex]
% > Roman page numbering
% ⧐ Table of Contents
% ⧐ Table of Figures
% ⧐ Table of Algorithms
\input{content/appendix/A/main.tex}
\input{content/appendix/B/main.tex}
\input{content/appendix/A/main.tex}
\input{content/appendix/B/main.tex}
\input{content/appendix/A/main.tex}
\input{content/appendix/B/main.tex}
\input{content/appendix/A/main.tex}
\input{content/appendix/B/main.tex}
% ~ ———————————————————————————————————

% ~ CONTENT ———————————————————————————
% > arabic page numbering
% $ CHAPTERS
% * ————— CHAPTER 1 —————
\input{content/appendix/A/main.tex}
\input{content/appendix/B/main.tex}
\input{content/appendix/A/main.tex}
\input{content/appendix/B/main.tex}
\input{content/appendix/A/main.tex}
\input{content/appendix/B/main.tex}
\input{content/appendix/A/main.tex}
\input{content/appendix/B/main.tex}
% * ————— CHAPTER 2 —————
\input{content/appendix/A/main.tex}
\input{content/appendix/B/main.tex}
\input{content/appendix/A/main.tex}
\input{content/appendix/B/main.tex}
\input{content/appendix/A/main.tex}
\input{content/appendix/B/main.tex}
\input{content/appendix/A/main.tex}
\input{content/appendix/B/main.tex}
% * ————— CHAPTER 3 —————
% \input{content/appendix/A/main.tex}
\input{content/appendix/B/main.tex}
\input{content/appendix/A/main.tex}
\input{content/appendix/B/main.tex}
\input{content/appendix/A/main.tex}
\input{content/appendix/B/main.tex}
\input{content/appendix/A/main.tex}
\input{content/appendix/B/main.tex}
% ~ ———————————————————————————————————
\chapter{Conclusion}

L'étude de la méthodologie liée à la régression non paramétrique pour le modèle additif ainsi que le modèle semi-paramétrique permet de découvrir des outils qui peuvent s'avérer utiles à la fois pour la compréhension de la théorie de l'estimation non paramétrique mais aussi pour la mise en place d'algorithmes d'estimation. 

\bigskip

Parmis les idées les plus intéressantes, on peut citer l'utilisation du domaine fréquentiel et de la transformée de Fourier pour la déconvolution de l'estimation. Une des idées les plus marquantes doit être le point de vue de la régression non paramétrique du modèle additif comme une projection de la fonction $m$ sur un sous espace des fonctions additives, à la fois très géométrique et pour autant clé et non purement esthétique dans l'étude de la convergence. Enfin le cheminement qui mène à la considération du noyau normalisé de déconvolution pour le modèle semi-paramétrique comme l'assemblage de briques fondatrices qui proviennent des multiples éléments méthodologiques abordés dès le début de ce rapport est très satisfaisant.

\bigskip

Le stage a aussi permis de rendre compte de la difficulté de généraliser des concepts à des espaces plus généraux tels que les espaces de Banach, et les éléments auxquels il faut faire attention lorsque l'on manipule des données fonctionnelles avec un exemple concret de difficulté qui a dû être contournée pour pouvoir appliquer la méthodologie bien connue dans le cadre des réels.

% ~ APPENDIX ——————————————————————————
\appendix
\pagenumbering{roman}
% * ————— APPENDIX —————
\input{content/appendix/A/main.tex}
\input{content/appendix/B/main.tex}
\input{content/appendix/A/main.tex}
\input{content/appendix/B/main.tex}
\input{content/appendix/A/main.tex}
\input{content/appendix/B/main.tex}
\input{content/appendix/A/main.tex}
\input{content/appendix/B/main.tex}
% ~ ——————————————————————————————————

\typeout{}
\bibliography{bibliography/main}

\end{document}

