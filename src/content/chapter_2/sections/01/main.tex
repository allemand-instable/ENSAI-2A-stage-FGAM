Les données fonctionnelles constituent un modèle intéressant par leur capacité à modéliser le fait que \emph{la relation entre une réponse et des paramètres} (par exemple la pression d'un gaz parfait en fonction de la température ou la débit d'un fleuve en fonction de la position $(x,y,z)$ dans l'espace) est elle même sujette à une loi. Un exemple serait la puissance électrique instantanée consommée par un ménage à un instant $t$ de la journée, si chacun possède sa consommation propre, les ménages ont tendance à consommer plus ou moins de la même manière (moins dans la journée, plus le soir). Elles sont de plus en plus utilisées pour les données de capteur et dans l'industrie notamment. Si il existe déjà des outils pour faire de la régression fonctionnelle paramétrique, notamment avec le modèle linéaire, on voudrait pouvoir bénéficier de la flexibilité des modèles non paramétriques. Nous allons étudier dans cette section la généralisation des modèles additifs aux données fonctionnelles et les difficultés qui découlent d'une telle tentative.