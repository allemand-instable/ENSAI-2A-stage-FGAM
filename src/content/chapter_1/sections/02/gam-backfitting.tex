{

\question{ Comment estimer le modèle additif ? }

\subsubsection{Formulation Gauss-Sneidel}
    
    Alors on peut ré-écrire le système d'équations conditionnelles comme un système d'équations linéaires :

    \begin{equation}
        \begin{cases}
            m_1(X_1) + \mathit E_1 [m_2(X_2)] + \dots + \mathit E_1[m_d(X_d)]  &= \mathit E_1[Y] \\ 
            \hspace{4cm} \vdots &  \hspace{0.8cm} \vdots \\
            \mathit E_d [m_1(X_1)]  + \dots + \mathit E_d [m_{d-1}(X_{d-1})]  + m_d(X_d) &= \mathit E_d[Y]
        \end{cases}\iff \boxed{\mathbf E \circ \grandM(X) = E \circ Y}
    \end{equation}

    avec la matrice $\mathbf E$, l'application $E$ et le vecteur $\grandM(X)$ définis par :

    \begin{minipage}{0.33\textwidth}
        \begin{equation}
            \mathbf E =
            \begin{bmatrix}
                I           & \mathit E_1 & \dots  & \dots  & \mathit E_1 \\
                \mathit E_2 & I           & E_2    & \dots  & \mathit E_2 \\
                \mathit E_3 & \mathit E_3 & \ddots &        & \mathit E_3 \\
                \vdots      & \vdots      &        & \ddots & \vdots      \\
                \mathit E_d & \mathit E_d & \dots  & \dots  & I
            \end{bmatrix}
            \label{eq:gam-matrix-E}
        \end{equation}
    \end{minipage}
    \begin{minipage}{0.35\textwidth}
        \begin{equation}
            E : \func{\VA{\grandR}}{\VA{\R d}}{Y}{\begin{bmatrix}
                \mathit E_1[Y] \\
                \vdots \\
                \mathit E_d[Y]
            \end{bmatrix}}
        \end{equation}
    \end{minipage}
    \begin{minipage}{0.25\textwidth}
        \begin{equation}
            \grandM(X) =
            \begin{bmatrix}
                m_1(X_1) \\
                m_2(X_2) \\
                \vdots   \\
                m_d(X_d)
            \end{bmatrix}
        \end{equation}
    \end{minipage}

}
