{
    L'idée du modèle additif est de combattre le fléau de la dimension en ajoutant une hypothèse que l'on espère peu coûteuse sur la fonction à estimer : la fonction $m$ est supposée être la somme de $d$ fonctions $m_j$ à une dimension. On peut alors espérer avoir la vitesse de convergence de la régression non paramétrique de la dimension $1$ en effectuant $d$ régressions non paramétriques à une dimension.

    \begin{equation}
        \begin{cases}
            X = \begin{bmatrix}
                X_1\\
                \vdots\\
                X_d
            \end{bmatrix}
            \\\\
            m(X) = \sum_{j = 1}^d m_j(X_j)
        \end{cases}
    \end{equation}


}

