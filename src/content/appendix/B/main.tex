\chapter{some code}

\section*{données fonctionnelles pour le praticien}

\ifnum\value{code}=1
	\begin{minted}[linenos=true, mathescape=true, frame=single, breaklines]{R}
    # ——— install ——— #
    install.packages(c("fda", "fda.usc"))
    # ——— general packages ——— #
    library(data.table)
    # ——— FDA packages ——— #
    library(fda)
    library(fda.usc)
\end{minted}
\fi

\ifnum\value{code}=1
	\begin{minted}[linenos=true, mathescape, frame=single, breaklines]{R} 
# |     date      | $X_1$ | $X_2$ | $\cdots$ | $X_p$ |
# | Jan 1st 12:00 | $\vdots$   | $\vdots$   |     | $\vdots$   |
data <- fread("data.csv")


# un individu = une ligne
# donc pour une série temporelle, il faut transposer les observations et avoir la suite des données disposées sur une ligne.
fdata_standard_index <- fda.usc::fdata(
    mdata = t(X),
    argvals = to_unit_interval(
    #               $\uparrow$
    # on doit ramener les dates dans l'intervalle $[0,1]$
        data[, .(date)]
    )
)
\end{minted}
\fi

\ifnum\value{code}=1
	\begin{minted}[linenos=true, mathescape, frame=single, breaklines]{R}
nb_points <- ncol(fdata)
nb_ts <- nrow(fdata)
 
fda_optim_basis <- fda.usc::optim.basis(
    fdataobj = select_representative_observations_for_mean_function_fdata(fdata_ts = fdata, is_iid = is_iid),
    type.CV = fda.usc::GCV.S,
    W = NULL,
    lambda = lambda_CV_look_list,
    numbasis = num_basis__seq,
    type.basis = "bspline",
    verbose = TRUE
)
\end{minted}
\fi

\ifnum\value{code}=1
	\begin{minted}[linenos=true, mathescape, frame=single, breaklines]{R}
fda_optimal_basis <- ...
fdata_obj_temp <- fda_optimal_basis[["fdata.est"]]
fdata_obj <- fda.usc::fdata2fd(fdata_obj_temp)
fpca_result <- fda::pca.fd(
    fdobj = fdata_obj,
    nharm = 3,
    # centrer les données
    centerfns = TRUE
)
\end{minted}
\fi

Regardons désormais à quoi ressemble la sortie :

\ifnum\value{code}=1
	\mintinline{R}{fpca_result$scores} $ = \quad
		\begin{array}{cc}
			                & \longrightarrow[\phi_k] \\
			\downarrow[X_i] &
			\begin{bmatrix}
				\ddots & \cdots                                           & \vdots \\
				\vdots & \xi_i^{[k]} = \langle X_i - \mu | \phi_k \rangle & \vdots \\
				\cdots & \cdots                                           & \ddots \\
			\end{bmatrix}
		\end{array}
	$
\fi

\ifnum\value{code}=1
	\begin{minted}[linenos=true, mathescape, frame=single, breaklines]{R}
list(
        fpca_result = fpca_result,
        scores = fpca_result$scores,
        eigenfunctions = fpca_result$harmonics,
        explained_variance = fpca_result$varprop
    )
\end{minted}
\fi
