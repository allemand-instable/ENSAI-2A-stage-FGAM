\begin{tabularx}{\linewidth}{lX}
	\toprule
	\textbf{Notation}  & \textbf{Signification} \\
	%
	\midrule
	\textbf{Probabilités} &                        \\
	\midrule

	$\leb$ ou $\displaystyle\int$ & Intégrale de Lebesgue \\
	$\boch$ & Intégrale de Böchner \\

    \midrule
	\textbf{Statistiques} &                        \\
	\midrule

	$\VA{E}$ & ensemble des Variables Aléatoires à valeurs dans un ensemble $E$ : applications mesurables $(\Omega, \calF, \P) \rightarrow E$ \\

	$X$ & Variable Aléatoire\\
	$\widehat{X}$ & Quantité empirique \\
	$\widetilde{X}$ & Quantité intangible (\og unfeasible \fg dans la littérature)\\
	$\overline{X}$ & Moyenne empirique\\
	$X_d^{[ \, i \,]}$ & $d^{eme}$ composante de l'observation $i$ de la variable aléatoire multivariée $X$\\



	\midrule
	\textbf{Algèbre Linéaire} &                        \\
	\midrule

	$\mathit{E_k}$ & Opérateur d'espérance conditionnelle selon la composante $k$ : $$\mathit E_k : \func{\VA{E^d}}{\VA{E}}{g(X)}{\esperancesachant{X_k}{g(X)}}$$ \\
	% TODO : ajouter ref
	$\mathbf E$ $\left(\underset {\textsf{\faExclamationTriangle}} {\neq} \E\right)$& Matrice du système d'équation conditionné du modèle additif ($cf$ \ref{eq:gam-matrix-E} )\\



	\midrule
	\textbf{Ensembles} &                        \\
	\midrule
\end{tabularx}