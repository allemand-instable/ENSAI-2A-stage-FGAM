Pendant le cursus 2A réalisé à l'ENSAI, les étudiants découvrent la régression non paramétrique qui permet d'estimer la loi conditionnelle d'une réponse vis à vis de covariables sans hypothèse sur la forme de la relation entre la réponse et les covariables. On peut alors complexifier les modèles de données que l'on considère en tirant avantage des bénéfices de l'approche paramétrique, aux vitesses de convergence rapide des estimateurs, et de l'approche non paramétrique, flexible et plus robuste à l'erreur de choix du modèle. On appelle cela une méthode \og semi-paramétrique \fg. Un modèle semi-paramétrique courant est le modèle partiellement linéaire avec la composante non paramétrique supposée additive. L'ensemble des concepts et des motivations sont introduites dans ce stage.

\bigskip

Il existe un modèle de données appelées données fonctionnelles qui sont de plus en plus présentes dans différents champs d'application de la statistique : santé, sport, industrie \ldots. De la théorie a déjà été produite sur la régression fonctionnelle. 
Peut-on faire de la régression semi-paramétrique fonctionnelle ? Le sujet de ce stage est l'étude à partir de diverses ressources bibliographiques sur la régression semi-paramétrique, de comprendre dans un premier lieu à partir du savoir d'un étudiant de 2$^e$ année du cursus ingénieur de l'ENSAI la méthodologie derrière la régression semi-paramétrique du modèle partiellement-linéaire dans le cadre réel. Enfin on pourra rendre compte des difficultés rencontrées ainsi que les différences et similarités dans les approches lors de l'extension des méthodes du modèle additif au cadre fonctionnel.